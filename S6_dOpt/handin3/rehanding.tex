% Created 2018-04-22 s� 13:27
\documentclass[11pt]{article}
\usepackage[utf8]{inputenc}
\usepackage[T1]{fontenc}
\usepackage{fixltx2e}
\usepackage{graphicx}
\usepackage{longtable}
\usepackage{float}
\usepackage{wrapfig}
\usepackage{rotating}
\usepackage[normalem]{ulem}
\usepackage{amsmath}
\usepackage{textcomp}
\usepackage{marvosym}
\usepackage{wasysym}
\usepackage{amssymb}
\usepackage{hyperref}
\tolerance=1000
\usepackage{color}
\usepackage{listings}
\usepackage{bm}
\author{Christian Zhuang-Qing Nielsen\thanks{201504624, christian@czn.dk}}
\date{\today}
\title{dOpt Compulsory Assignment 3 Re-handin}
\hypersetup{
  pdfkeywords={},
  pdfsubject={},
  pdfcreator={Emacs 25.1.1 (Org mode 8.2.10)}}
\begin{document}

\maketitle
\tableofcontents


\section{Proof}
\label{sec-1}
We want to prove that one of two languages are in $\bm{P}$ iff the other one is as well. These languages are decided by the same function, but the input to the deciding function is represented differently (both are still strings $x \in \{0, 1\}^*$) and the two representations are \emph{good} and \emph{polynomially equivalent}.  \\ \\
Starting off, we can assume that one of the languages, lets say $L_1$, is in $\bm{P}$:
\begin{equation*}
$L_1 \in \bm{P}$
\end{equation}
Under this assumption, the language is decided by a Turing machine $T_1$ in polynomial time. Now to continue our proof we need to show that $L_2$ is also in $\bm{P}$. From our assumption, we know that $\pi_1$ is a good representation and we want to map it to $\pi_2$. However, we cannot assume that $x \in \pi_2(S)$ just because they are polynomially equivalent.  \\ \\
Since $\pi_2$ is also supposedly a good representation we can create a Turing machine $T_2$ that decides the language it represents in polynomial time. Being certain of a good representation AND knowing it is polynomially equivalent with $pi_1$ from the description, we can perform the mapping 
\begin{equation*}
$\pi_1(x) = r_1(\pi_2(x))$.
\end{equation}
Since every input to the mapping was a good representation (and thus decidable by a Turing machine) and it can be mapped back to $\pi_1$ used by $L_1$, which was definitely in $\bm{P}$ from our assumption, which translates back to $L_2 \in \bm{P}$  iff it is decided by a TM $T_2$ and $L_1$ is decided by a TM $T_1$. \square

\section{Notes}
\label{sec-2}
So the idea is that  if $L_1 \in \bm{P}$ we can utilise a mapping between the polynomially equivalent representations. We know that $\pi_1$ is a good representation from our assumption (and our constructed Turing machine). For the mapping to make sense, $\pi_2$ must also be a good representation, which is why we construct the second Turing machine that hopefully decides the language $L_2$. This can be used to confirm that $L_2 \in \bm{P}$ if and only if our  assumptions are true (i.e. that $\pi_2$ is a good representation and that $L_1 \in \bm{P}$). 

This was how I interpreted the proof made using Turing machines.
% Emacs 25.1.1 (Org mode 8.2.10)
\end{document}
